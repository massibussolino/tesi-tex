\chapter{Conclusion}

This thesis presents a study on relative navigation around a known uncooperative target using multispectral imaging sensors. A novel visual navigation pipeline has been developed to exploit images acquired by both a visible and a thermal monocular camera, either by fusing the information or using a single spectrum. The proposed navigation solution has been critically tested on synthetic images of the VESPA debris to evaluate its performance and limitations.\\
In favorable illumination and thermal conditions, the multispectral data fusion resulted in an enhancement of the pose estimation performance at the expense of a mild increase in computational time. This results demonstrates the proper working of the sensor fusion using a tightly coupled filtering approach applied to the navigation problem, as the EKF successfully fuses the sensor's data providing an improved estimate of the pose. The effectiveness of multispectral data fusion is also enhanced by the adaptivity of the measurement noise matrix, which balances the contribution of the two sensors given their different accuracy.\\
However, when tested in low illumination conditions, the navigation pipeline was unable to track the relative position and attitude, as the estimated pose consistently diverged from the ground truth. It can be concluded that with the proposed navigation algorithm, the multispectral approach does not provide a reliable standalone solution when different illumination conditions are encountered. The limiting factor is identified as the visible image, which degrades the performance of the Image Processing pipeline. As a consequence, the possibility of exploiting thermal-only navigation when the visible sensor is compromised has been investigated.\\
The TIR-only navigation application has been tested in both the hot and cold cases of the target. The main difference between the two situations is that in the hot case, the target’s temperature is fully within the thermal sensor’s temperature range, providing a vivid image. In contrast, in the cold case, the target appears less distinct in the thermal image since its temperature is closer to the lower limit of the sensor’s detection range. Although the results are promising, they are not satisfactory enough to confirm thermal-only navigation as a reliable solution. In the hot case, a divergence behavior was identified at the end of the simulations, while this behavior was prominent from the early stages in the cold case. However, the error drift is mainly caused by the target’s symmetry associated with the low performance of Image Processing using only thermal images, which is not able to successfully track the superficial elements of VESPA. This indicates that, with a more distinguishable target shape, thermal navigation might indeed provide robust performances.\\
% From the obtained results, it is assessed that multispectral data fusion does not directly increase the applicability of monocular navigation for visible cameras. Instead, its contribution is limited to enhancing the accuracy of pose estimation. In cases where the target is not fully visible, it proved more convenient to discard the visible image entirely and use only the output from the thermal sensor. Although TIR-only navigation is not yet mature enough to provide consistently accurate tracking of the chaser-target relative position, the obtained results are promising and suggest that further research in this area could yield significant improvements.\\
From the obtained results, the original research questions can be answered as:
\begin{enumerate}
    \item Multispectral imaging can improve the relative navigation about a known non-cooperative target. The thermal imaging contribution is limited to an improvement of the performances in good illumination conditions, while it does not successfully increase the range of applications of visible navigation because of the  combined degradation of the visible information and the low robustness of the IP routine applied to the thermal images.  
    %However, the results are promising and suggest that further research might increase the maturity of thermal-based methods to provide an extension to the environmental conditions for which monocular cameras can be used.
    \item The most effective imaging measurements management method is to use different spectral modalities in different environmental conditions. If in good illumination conditions the multispectral data fusion increases the navigation performances, in harsher conditions it is beneficial to discard the visible measurements. This highlights an environmental triggered switch between different modalities as the best solution.
    \item From the obtained results, thermal sensors can not provide a standalone solution to the relative navigation problem because the low performance of the Image Processing routine applied to thermal images. However, as the limitations are also related to the particular shape of the VESPA target, it is believed that with a more distinguishable target thermal-only navigation might produce valuable results.
    \item A tightly coupled sensor fusion approach successfully solves the relative navigation problem. However, because of the dependency of the results on the test case, it is not possible to quantitatively compare the obtained results with other works. It is therefore not identified at which level it is more convenient to fuse the multispectral data. 
\end{enumerate}
Even if the validation methodology presents a solid basis, the overall lack of realism in the testing framework may impact the validity of the results. For instance, the absence of noise in the images and background elements creates favorable conditions. However, the simplified structural and thermal model used to generate thermal images introduces fictitious elements not expected in real imagery. To provide a more insightful evaluation of the navigation pipeline, it would be beneficial to design a more realistic testing framework. 

\section{Future work}
\chaptermark{Conclusion}

Because of its relevance for next-generation space missions, autonomous relative navigation about uncooperative target is a continuously evolving subject in the research field. With the growing interest in multispectral navigation in recent years, the obtained results in this thesis work represent a preliminary but promising starting point for further advancement in this area. Some possible future developments identified during the thesis work are here provided:

\subsection*{Testing framework refinement}
As mentioned in the previous paragraph, more realistic synthetic images should be implemented to achieve a more insightful assessment of the navigation pipeline’s performance. Since both visible and thermal images are currently rendered without noise, it should be added in post-processing, as it may not be negligible for the application considered \cite{BECHINI2023358}. An important simplification of the current testing framework is the use of a simplified thermal model for the target. For future studies, it is suggested to increase the realism of the thermal model by considering transient behaviors and a non-uniform temperature profile of the target. In this context, the work proposed by \cite{CIVARDI2023} presents a refined thermal rendering tool tailored for space imagery.

\subsection*{Multispectral navigation}
In this work, the data fusion has been performed at feature level, demonstrating the possibility to perform multispectral data fusion with a tightly coupled approach. 
However, the results suggest that the proposed multispectral algorithm provides a meaningful solution only when the target is clearly visible in both thermal and visible images. As one proposed approach to tackle this issue involves using thermal navigation as a standalone solution for low illumination conditions, further effort should be focused on making this method more reliable. One advantage of the proposed solution is its flexibility to changes in the Image Processing routine, enabling the investigation of hybrid approaches for the visible and thermal spectrum. An IP routine tailored specifically for thermal images or using more innovative techniques, such as CNNs, could provide an important contribution to the applicability of the proposed navigation filter.

\subsection*{Extension to different mission scenarios}
The visual navigation pipeline developed has been presented and tested in an ADR scenario. However, it can be adjusted for other space missions involving navigation around a known target, such as OOS or FF. As the navigation filter provided promising results for a peculiar target such as VESPA, it is expected that testing the pipeline on targets with more distinguishable shapes will yield even better results than those presented in the thesis. Moreover, the flexibility of the tightly coupled approach allows for straightforward integration of additional instruments in the sensor suite, making it easy to adapt for cooperative relative navigation. \\
The navigation solution could also be adapted for small-body exploration, assuming either a known asteroid or successful mapping of it. In this context, it would be interesting to compare this work with that proposed in \cite{civardi2021small}, where a loosely coupled approach was used to navigate around the Ryugu asteroid.

\clearpage
\thispagestyle{empty}