
\chapter*{Sommario in italiano}
\thispagestyle{empty}

\lettrine{L}{e} operazioni di prossimità ravvicinata ricoprono un ruolo cruciale nelle nuove classi di missioni, come la Rimozione Attiva di Detriti (ADR). Avvicinandosi ad un target non cooperativo, l'elevato rischio di collisione e la scarsa possibilità di affidarsi ad un intervento da terra richiedono la stima autonoma della posa (assetto e posizione) dell' inseguitore. Le fotocamere monoculari nello spettro visibile sono state ampiamente studiate e testate per applicazioni di navigazione visto il loro bilanciato rapporto tra la qualità dei dati che forniscono e le ridotte massa e potenza richieste. Tuttavia, la forte dipendenza dei dati forniti dalla condizione di illuminazione del target riduce fortemente le possibilità di applicazione del sensore. Una soluzione frequentemente proposta per mitigare questo problema è l'implementazione di una fotocamera monoculare termica insieme a quella visibile. I sensori termici forniscono dati di minor qualità rispetto a quelli visibili, ma sono meno dipendenti dalle condizioni di illuminazione.
Questa tesi propone una catena di navigazione innovativa per effettuare il rilevamento della posa tramite misurazioni ottenute da una camera visibile e una termica. L'algoritmo di navigazione usa i punti d'interesse estratti dalle immagini per effettuare la stima della posa con un approccio model-based rispetto ad un obiettivo noto. I punti d'interesse identificati nelle immagini multispettrali sono fusi con un approccio strettamente accoppiato attraverso un Filtro di Kalman Esteso (EKF), il cui output è la posa corretta dell' inseguitore. La tesi presenta il suddetto algoritmo, che effettua con successo la stima della posa, insieme ai test effettuati su immagini sintetiche. Si valuta il contributo della combinazione delle fotocamere visibili e termiche, analizzando inoltre la possibile applicazione della navigazione solo termica in scarse condizioni di illuminazione. La tesi vuole contribuire alla comprensione delle prestazioni e della robustezza della navigazione ottica multispettrale e a sviluppare algoritmi per la stima della posa che possano operare in ambienti difficili come quello della Rimozione Attiva di Detriti.
\\ \mbox{}\\ \noindent
\textbf{Parole chiave:} Operazioni di prossimità, Navigazione relativa ottica,  Stima non lineare, Fusione di dati multispettrali