

\chapter*{Abstract}
\thispagestyle{empty}
% \lettrine{C}{lose}-proximity operations play a crucial role in new classes of missions, such as Active Debris Removal. Approaching a non-cooperative target, the elevated risk of collision and the scarce reliance on ground intervention dictate the need for autonomous onboard pose (attitude and position) estimation for the chaser spacecraft. Monocular cameras operating in the visible (VIS) spectrum have been widely investigated and in-flight tested for navigation applications because of their adequate balance between
% the quality of the data provided and the limited mass and power consumption. However, the strong dependency of the results on the target’s illumination condition strongly reduces this sensor’s application range. A frequently investigated solution to mitigate this problem is the implementation of a monocular thermal-infrared (TIR) camera concurrently with the visible one. Thermal sensors provide lower-quality data than visible sensors but are less dependent on illumination conditions. \\
% This thesis proposes a novel visual navigation pipeline to perform pose tracking with measurements obtained from a VIS and a TIR monocular camera. The objectives are to assess the improvement in accuracy fusing the information extracted from the multispectral images and to evaluate the navigation algorithm’s performances in harsh illumination conditions where visible measurements are unavailable. The proposed visual navigation algorithm exploits the point features extracted from the images to perform model-based pose estimation with respect to a known target. The features extracted from the multispectral images are fused with a tightly coupled approach through an Extended Kalman Filter, whose output is the refined pose of the chaser. Such innovative pipeline will be presented in detail in the thesis, along with tests performed with synthetic
% images. \\ 
% The proposed navigation pipeline successfully performs the pose estimation: results will be presented, assessing the contribution of combining a TIR and a VIS monocular camera, studying the performance of the fused multispectral information and characterizing the performance of TIR-only navigation in low-illumination conditions. A comparative review is performed with techniques which exploit multispectral image-fusion methods to benchmark different approaches to this problem. This thesis is intended to contribute to the assessment of the performance and robustness of multispectral visual navigation and developing pose estimation schemes that operate in demanding
% environments like active debris removal. 



\lettrine{C}{lose}-proximity operations play a crucial role in new classes of missions, such as Active Debris Removal. Approaching a non-cooperative target, the elevated risk of collision and the scarce reliance on ground intervention dictate the need for autonomous onboard pose (attitude and position) estimation for the chaser spacecraft. Monocular cameras operating in the visible (VIS) spectrum have been widely investigated and in-flight tested for navigation applications because of their adequate balance between the quality of the provided data and the limited mass and power consumption. However, the strong dependency of the results on the target’s illumination condition strongly reduces this sensor’s application range. A frequently investigated solution to mitigate this problem is the implementation of a monocular thermal-infrared (TIR) camera concurrently with the visible one. Thermal sensors provide lower-quality data than visible sensors but they are less dependent on illumination conditions.
This thesis proposes a novel visual navigation pipeline to perform pose tracking with measurements obtained from a VIS and a TIR monocular camera. The navigation algorithm exploits the point features extracted from the images to perform model-based pose estimation with respect to a known target. The point features detected in the multispectral images are fused with a tightly coupled approach through an Extended Kalman Filter, whose output is the refined pose of the chaser. Such innovative pipeline will be presented in detail in this thesis, along with tests performed with synthetic images.
The proposed navigation chain successfully performs the pose estimation: results will be presented, assessing the contribution of combining a TIR and a VIS monocular camera, studying the performance of the fused multispectral information and characterizing the performance of TIR-only navigation in low-illumination conditions.  This thesis is intended to contribute to the assessment of the performance and robustness of multispectral visual navigation and developing pose estimation schemes which operate in demanding environments like Active Debris Removal.
\\ \mbox{}\\ \noindent
\textbf{Keywords:} Proximity operations, Visual-based relative navigation, Non-linear estimation, Multispectral data fusion

